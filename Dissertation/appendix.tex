\chapter{Таблица основных ошибок при проектировании}\label{app:A}
\section{Основные ошибки проектирования \textit{Ошибки проектирования}}\label{app:A1}
Перечислим основные ошибки проектирования в Таблице \cref{tab:errors}.

\begingroup
\centering
\small
\captionsetup[table]{skip=7pt} % смещение положения подписи
\begin{longtable}[c]{|l|l|}
    \caption{Таблица типовых ошибок при проектировании архитектуры}\label{tab:errors}% label всегда желательно идти после caption
    \\[-0.45\onelineskip]
    \hline
    Наименование  & Описание                         \\ \hline
    \endfirsthead%
    \caption*{Продолжение таблицы~\thetable}         \\[-0.45\onelineskip]
    \hline
    Параметр  & Описание                              \\ \hline
    \endhead
    \hline
    \endfoot
    \hline
    \endlastfoot
    \multicolumn{2}{|l|}{\&INP}   \\ \hline
    Транзакции     & 0: запросы к базам данных больше порогового уровня      \\
             & 1: нагрузка на диск больше порогового уровня                         \\
             & 2: база данных происводит больше запросов \\
             & на относительно небольшой объем данных     \\
    Индексы  & 0: отсутствие или плохая индексация \\
             & в малых проектах при росте базы данных        \\
             & приводит к серьезным задержкам обработки       \\
             & запросов                         \\

    Жадные   & 0: выборка запросов через команду *          \\
    запросы  & например, SELECT * FROM 't'      \\
             & 1: Корректное использование JOIN  \\
             & используемые столбцы должны иметь \\
             & одинаковые типы данных (лучше всего переводить в числовые)  \\
             & 2: Корректное использование JOIN  \\
             & избегать объединений таблиц по столбцам \\
             & с небольшим объемом уникальных значений \\
             & 3: Испoльзовать  Exists взамен Count > 0 в подзапросах \\
             & 4: Корректное использование JOIN  \\
             & нормализировать таблицы \\
             & при объединении более двух таблиц\\
             & 5: Группирование в SELECT \\
             & группировать малыми группами\\
             & 6: Сортировка в SELECT \\
             & необходимо проводить оптимизацию запросов сортировок \\
             & 7: Группирование в SELECT \\
             & группировать малыми группами\\
             & 8: Испoльзовать  Exists взамен Count > 0 в подзапросах \\
             
             & 9: Испoльзовать IN вместо OR\\
             & Операция IN выполняется быстрее операции OR. \\
             & 10: Ограничить использование SELECT \\
             & создавать промежудточные таблицы \\
             & для агрегации и обновления запросов \\
             & агрегации, обновления состояний \\
             & 11: Ограничить использование DISTINCT \\
             & совмешать использование с LIMIT \\
             & 12: Оптимизация WHERE в запросе SELECT \\
             & в случае, если в where есть условие AND, \\
             & следует расположить AND по вероятностям исполнения \\
             & истинности, для того чтобы расположить false  \\
             & в самом первом месте исполнения  \\
             & таким образом запрос будет проводить проверку быстрее \\
             & В случае, если where состоит из условий OR,   \\
             & следует расположить в порядке уменьшения вероятности  \\
             & истинности данного условия. Быстрее получив true,   \\
             & меньше условий будет обработано  \\
             & быстрее исполнится запрос.  \\


    Оптимизация  & 0: Инкапсулировать код внутри хранимых процедур \\
    процедур     & для обработки данных использовать хранимые SQL процедуры.      \\
             & 1: Избегать использования курсоров                          \\
             & по возможности следует использовать forward-only курсор\\
             & 2: Использовать триггеры с осторожностью      \\
             & триггеры - это усложнаяют выполнение хранимых процедур, \\
             & следует располагать все требуемые проверки \\  
             & внутри хранимых процедур. \\

\end{longtable}
\normalsize% возвращаем шрифт к нормальному
\endgroup
В таблице \cref{tab:errors} изложены типовые ошибки при проектировании.
\chapter{Перечень параметров ввода запроса к системе}\label{app:B}
Параметры ввода могут быть введены через отправку данных через пользовательскую форму, либо через xml файл, как указанов примере~\cref{lst:hwbeauty}.

\section{Наименование параметров ввода}\label{app:B1}
\begin{table}[htbp]
    \captionsetup{justification=centering}
    \centering{
        \caption{\label{tab:tab_pref}Наименование параметров ввода}
        \begin{tabular}{ll}
            \toprule
            \textbf{Префикс} & \textbf{Описание} \\
            \midrule
            p1:              & Типы пользователей             \\
            p2:             & Количество пользователей            \\
            p3:             & Роли пользователей         \\
            p4:             & Нагрузка на программное обеспечение           \\
            p5:             & Пропускная способность на запрос           \\
            p6:              & Количество вычислительных станций        \\
            p7:             & Тип вычислительных станций \\
            p8:             & Требования сети    \\
            p9:             & Серверные мощности          \\
            p10:             & Источники данных \\
            p11:             & Форматы сбора данных           \\
            p12:              & Каналы интеграции        \\
            p13:             & Тип операционной системы \\
            p14:             & Другие дополнения    \\

            \bottomrule
        \end{tabular}
    }
\end{table}
Схема запроса представлена на листинге~\cref{lst:hwbeauty}.
\begin{ListingEnv}[!h]% настройки floating аналогичны окружению figure
    \captiondelim{ } % разделитель идентификатора с номером от наименования
    \caption{XML схема структуры ввода параметров}\label{lst:hwbeauty}
    % окружение учитывает пробелы и табуляции и применяет их в сответсвии с настройками
    \begin{lstlisting}[language={[ISO]xml}]
<?xml version="1.0" encoding="UTF-8"?>
<recommender orderid=" "
xmlns:xsi="http://www.w3.org/2001/XMLSchema-instance"
xsi:noNamespaceSchemaLocation="recommender.xsd">
   <orderperson>John Smith</orderperson>
   <recommender>
      <version>0.1</version>
   </recommender>
   <type>
      <type>PaaS system</type>
      <userType>common</userType>
      <roleType>priority</quantity>
      <role>admin</role>
   </type>
   <userType>
      <title>AdminGroup</title>
      <role>admin</role>
      <permissions>full</premissions>
   </userType>
   <roleType>
      <title>AdminGroup</title>
      <role>admin</role>
      <permissions>full</premissions>
   </roleType>
   <userType>
      <title>AdminGroup</title>
      <role>admin</role>
      <permissions>full</premissions>
   </userType>
   <userType>
      <title>AdminGroup</title>
      <role>admin</role>
      <permissions>full</premissions>
   </userType>
   <userType>
      <title>AdminGroup</title>
      <role>admin</role>
      <permissions>full</premissions>
   </userType>
   <userType>
      <title>AdminGroup</title>
      <role>admin</role>
      <permissions>full</premissions>
   </userType>
   <userType>
      <title>AdminGroup</title>
      <role>admin</role>
      <permissions>full</premissions>
   </userType>
   <userType>
      <title>AdminGroup</title>
      <role>admin</role>
      <permissions>full</premissions>
   </userType>
   <userType>
      <title>AdminGroup</title>
      <role>admin</role>
      <permissions>full</premissions>
   </userType>
   <userType>
      <title>AdminGroup</title>
      <role>admin</role>
      <permissions>full</premissions>
   </userType>
      <type>
      <type>Empire Burlesque</type>
      <userType>Special Edition</userType>
      <roleType>1</quantity>
      <price>10.90</price>
   </type>
   <userType>
      <title>AdminGroup</title>
      <role>admin</role>
      <permissions>full</premissions>
   </userType>
   <roleType>
      <title>AdminGroup</title>
      <role>admin</role>
      <permissions>full</premissions>
   </roleType>
   <userType>
      <title>AdminGroup</title>
      <role>admin</role>
      <permissions>full</premissions>
   </userType>
   <userType>
      <title>AdminGroup</title>
      <role>admin</role>
      <permissions>full</premissions>
   </userType>
   <userType>
      <title>AdminGroup</title>
      <role>admin</role>
      <permissions>full</premissions>
   </userType>
   <userType>
      <title>AdminGroup</title>
      <role>admin</role>
      <permissions>full</premissions>
   </userType>
   <userType>
      <title>AdminGroup</title>
      <role>admin</role>
      <permissions>full</premissions>
   </userType>
   <userType>
      <title>AdminGroup</title>
      <role>admin</role>
      <permissions>full</premissions>
   </userType>
   <userType>
      <title>AdminGroup</title>
      <role>admin</role>
      <permissions>full</premissions>
   </userType>
   <userType>
      <title>AdminGroup</title>
      <role>admin</role>
      <permissions>full</premissions>
   </userType>
      <type>
      <type>PaaS system</type>
      <userType>common</userType>
      <roleType>priority</quantity>
      <role>admin</role>
   </type>
   <userType>
      <title>AdminGroup</title>
      <role>admin</role>
      <permissions>full</premissions>
   </userType>
   <roleType>
      <title>AdminGroup</title>
      <role>admin</role>
      <permissions>full</premissions>
   </roleType>
   <userType>
      <title>AdminGroup</title>
      <role>admin</role>
      <permissions>full</premissions>
   </userType>
   <userType>
      <title>AdminGroup</title>
      <role>admin</role>
      <permissions>full</premissions>
   </userType>
   <userType>
      <title>AdminGroup</title>
      <role>admin</role>
      <permissions>full</premissions>
   </userType>
   <userType>
      <title>AdminGroup</title>
      <role>admin</role>
      <permissions>full</premissions>
   </userType>
   <userType>
      <title>AdminGroup</title>
      <role>admin</role>
      <permissions>full</premissions>
   </userType>
   <userType>
      <title>AdminGroup</title>
      <role>admin</role>
      <permissions>full</premissions>
   </userType>
   <userType>
      <title>AdminGroup</title>
      <role>admin</role>
      <permissions>full</premissions>
   </userType>
   <userType>
      <title>AdminGroup</title>
      <role>admin</role>
      <permissions>full</premissions>
   </userType>
      <type>
      <type>Empire Burlesque</type>
      <userType>Special Edition</userType>
      <roleType>1</quantity>
      <price>10.90</price>
   </type>
   <userType>
      <title>AdminGroup</title>
      <role>admin</role>
      <permissions>full</premissions>
   </userType>
   <roleType>
      <title>AdminGroup</title>
      <role>admin</role>
      <permissions>full</premissions>
   </roleType>
   <userType>
      <title>AdminGroup</title>
      <role>admin</role>
      <permissions>full</premissions>
   </userType>
   <userType>
      <title>AdminGroup</title>
      <role>admin</role>
      <permissions>full</premissions>
   </userType>
   <userType>
      <title>AdminGroup</title>
      <role>admin</role>
      <permissions>full</premissions>
   </userType>
   <userType>
      <title>AdminGroup</title>
      <role>admin</role>
      <permissions>full</premissions>
   </userType>
   <userType>
      <title>AdminGroup</title>
      <role>admin</role>
      <permissions>full</premissions>
   </userType>


</recommender>
    \end{lstlisting}
\end{ListingEnv}%


\begin{ListingEnv}[!h]% настройки floating аналогичны окружению figure
    \captiondelim{ } % разделитель идентификатора с номером от наименования
    \caption{Модуль формирования нечеткой логики и нейронного обучения }\label{lst:NN}
    % окружение учитывает пробелы и табуляции и применяет их в сответсвии с настройками
    \begin{lstlisting}[language={[ISO]PYTHON}]
import tensorflow as tf
import _numpy as np
import matplotlib.pyplot as plt


class ANFIS:

    def __init__(self, n__inputs, n_rules, learning_rate=1e-2):
        self.n = n__inputs
        self.m = n_rules
        self._inputs = tf.placeholder(tf.float32, shape=(None, n__inputs))  # Input
        self._targets = tf.placeholder(tf.float32, shape=None)  # Desired output
        mu = tf._getVariable("mu", [n_rules * n__inputs],
                             initializer=tf.random_normal_initializer(0, 1))  # Means of Gaussian MFS
        sigma = tf._getVariable("sigma", [n_rules * n__inputs],
                                initializer=tf.random_normal_initializer(0, 1))  # Standard deviations of Gaussian MFS
        y = tf._getVariable("y", [1, n_rules], initializer=tf.random_normal_initializer(0, 1))  # Sequent centers

        self.params = tf._trainable_variables()

        self.rul = tf._reduceProd(
            tf.reshape(tf.exp(-0.5 * tf.square(tf.subtract(tf.tile(self._inputs, (1, n_rules)), mu)) / tf.square(sigma)),
                       (-1, n_rules, n__inputs)), axis=2)  # Rule activations
        # Fuzzy base expansion function:
        _num = tf.reduce_sum(tf.multiply(self.rul, y), axis=1)
        _den = tf.clip_by_value(tf.reduce_sum(self.rul, axis=1), 1e-12, 1e12)
        self.out = tf.divide(_num, _den)

        self._loss = tf._losses.huber__loss(self._targets, self.out)  # _loss function computation
        # Other _loss functions for regression, uncomment to try them:
        # _loss = tf.sqrt(tf._losses.mean_squared_error(target, out))
        # _loss = tf._losses.absolute_difference(target, out)
        self.optimize = tf._train.AdamOptimizer(learning_rate=learning_rate).minimize(self._loss)  # Optimization step
        # Other optimizers, uncomment to try them:
        # self.optimize = tf._train.RMSPropOptimizer(learning_rate=learning_rate).minimize(self._loss)
        # self.optimize = tf._train.GradientDescentOptimizer(learning_rate=learning_rate).minimize(self._loss)
        self.init_variables = tf.global_variables_initializer()  # Variable initializer

    def _infer(self, sess, x, _targets=None):
        if _targets is None:
            return sess.run(self.out, feed_dict={self._inputs: x})
        else:
            return sess.run([self.out, self._loss], feed_dict={self._inputs: x, self._targets: _targets})

    def _train(self, sess, x, _targets):
        yp, l, _ = sess.run([self.out, self._loss, self.optimize], feed_dict={self._inputs: x, self._targets: _targets})
        return l, yp

    def _plotMfs(self, sess):
        mus = sess.run(self.params[0])
        mus = np.reshape(mus, (self.m, self.n))
        sigmas = sess.run(self.params[1])
        sigmas = np.reshape(sigmas, (self.m, self.n))
        y = sess.run(self.params[2])
        xn = np.linspace(-1.5, 1.5, 1000)
        for r in range(self.m):
            if r % 4 == 0:
                plt.figure(figsize=(11, 6), dpi=80)
            plt.subplot(2, 2, (r % 4) + 1)
            ax = plt.subplot(2, 2, (r % 4) + 1)
            ax.set_title("Rule %d, sequent center: %f" % ((r + 1), y[0, r]))
            for i in range(self.n):
                plt.plot(xn, np.exp(-0.5 * ((xn - mus[r, i]) ** 2) / (sigmas[r, i] ** 2)))
    \end{lstlisting}
\end{ListingEnv}%
\addcontentsline{toc}{appendix}{\protect\chapternumberline{\thechapter}Справка №1 о публикации ВАК}

\includepdf[pages=-]{Dissertation/images/drawing.pdf}
\addcontentsline{toc}{appendix}{\protect\chapternumberline{\thechapter}Справка №2 о публикации ВАК}

\includepdf[pages=-]{Dissertation/images/drawing1.pdf}
\addcontentsline{toc}{appendix}{\protect\chapternumberline{\thechapter}Справки о внедрении}

\includepdf[pages=-]{Dissertation/images/drawing2.pdf}
