\chapter*{Заключение}                       % Заголовок
\addcontentsline{toc}{chapter}{Заключение}  % Добавляем его в оглавление

%% Согласно ГОСТ Р 7.0.11-2011:
%% 5.3.3 В заключении диссертации излагают итоги выполненного исследования, рекомендации, перспективы дальнейшей разработки темы.
%% 9.2.3 В заключении автореферата диссертации излагают итоги данного исследования, рекомендации и перспективы дальнейшей разработки темы.
%% Поэтому имеет смысл сделать эту часть общей и загрузить из одного файла в автореферат и в диссертацию:

Разработана модель интеллектуального архитектурного проектирования и структурного проектирования человеко-машинных систем, предназначенных для автоматизации производства и интеллектуального обеспечения процессов управления и необходимой обработки данных в организационно-технологических и распределенных системах управления в различных сферах технологического производства и других областях человеческой деятельности .Представлена модель рекомендательной системы для проектирования различных типов архитектуры информационных систем, разработанная на основе нейронной сети и нечеткого продукционной декомпозиции. Метод имеет ряд преимуществ и недостатков. Поскольку отдельные системы рекомендаций на основе декомпозиции могут давать переменные результаты, потенциал для реализации таких моделей заключается в разнообразии предлагаемых рекомендаций. Модель обучается автоматически, т.е. модель обрабатывает данные и сама идентифицирует продукционные правила по представленным характеристикам на основе представленных знаний о типах архитектур для разных типов информационных систем, что является преимуществом для генерации знаний за счет возможность автоматизированного масштабирования базы знаний, недостатком является сложность формализации семантических и лингвистических термов на этапе выявления отношений между сущностями для проектирования организации кода и отношений между сущностями базы данных.

Основные результаты работы заключаются в следующем:
\begin{enumerate}
	\item проанализированы конкурентные решения на рынке,
	\item разработана модель проеткирования автоматизированных систем управоления на основе системы поддержки принятия решения,
	\item разработана методология поддержки принятия решений при проеткировании архитектуры программного обеспечения,
	\item разработана математическая модель системы поддержки принятия решений для проектирования автоматизированных систем управления.
\end{enumerate}
%% Согласно ГОСТ Р 7.0.11-2011:
%% 5.3.3 В заключении диссертации излагают итоги выполненного исследования, рекомендации, перспективы дальнейшей разработки темы.
%% 9.2.3 В заключении автореферата диссертации излагают итоги данного исследования, рекомендации и перспективы дальнейшей разработки темы.

Итоги исследования:
\begin{enumerate}
  \item На основе анализа работы нейронных сетей, экспертных систем разработана модель поддержки принятия решений о проектировании архитектуры различных типов автоиматизированных систем управления,
  \item Численные исследования показали, что модель устойчива при малых отклонениях в возмущениях, т.е. недообучение и переобучение просиходят только в условиях потери устойчивости работы нейронной сети и нечеткого логического вывода в условиях сильных отклонений при обучении,
  \item Математическое моделирование показало, что модель выявляет эвристические соответствия между лингвистическими термами на основе аппроксимации и функционального отоборажения принадлежности из семантического пространства в численное за счет поправки ошибки и вектора градиента, выступающего как адаптивная система поправки ошибки,
  \item Для выполнения поставленных задач был создан алгоритм системы поддержки принятия решений, проанализирован и структурирован набор пользовательских запросов для формализации типа информационной системы, выполнена работа по проектировани. и реализации системы поддержки принятия решений.
\end{enumerate}

Рекомендации:
\begin{enumerate}
  \item в ходе исследования не получилось спроектировать систему организации шаблонов проектирования кода программного обеспечения, т.к. существует бесчисленный набор комбинаторных комбинаций по организации сущностей в логике программного обеспечения, 
  \item логика организации внутренней иерархии сущностей крайне зависит от конкретной доменной области технологического внедрения программного обеспечения, что делает невозможным предположить некоторую типовую организацию сущностей,
  \item логика иерархии таблиц в базах данных такэе подвержена многочисленным вероятным возможным вариантам связывания отношениями между собой и аналогичтным образом определяется доменной структурой технологического внедрения программного обеспечния, что делает невозможным предугадать рациональную декомпозицию, без знаний целей проектирования системы.
\end{enumerate}

Таким образом, работу по проектированию модели поддержки принятия решений была направлена на решение проблем с проектированием архитектуры программного обеспечения, т.к. в этой области есть сформированные типовые наборы архитектур внутренней и внещней организации для типовых классов автоматизированных систем управления.

Перспективы такого направления включают увеличение базы знаний, обучение за счет пользователей системы, размещение модели во всеобзем доступе, сбор обратной информации, улучшение функционала отображения входных переменных в выходные.

Перспективы дальнейшей разработки темы заключаются в усовершенствовании модуля обработки семантического набора пользовательских описаний состояний параметров и перменных, ждя заполнения процесса обучения по прецедентам и продукционным правилам. Система должна дополнять выводимый вектор ответов по степени правдоподобия,  с возможностью дополнать введенные данные. Однако, такая стратегия требоует разработки дополнительного модули авторизации пользователей для предоствращения злонамеренного вывода из строя нейронной сети и увеличения отклонений в обучении. Поэтому, доступ к обучению системы должны иметь только проверенные эксперты, которые должны доказатьналичие знаний посредством выполнения тестовых заданий на платформе реализации системы. Таким образом, систем, может стать полезным инструментом распределнного хранения знаний о разработки программного обеспечения не только для автоматизированных систем управления, но и для других типов информационных систем. Набирая и дополняя модули таблиц фактов, модулей формирования продукционных правил, модулей аггрегации данных по параметрам через обучение от различных экспертов в различных областях разработки программного обеспечения дает перспективу реализации потенциала системы, как полноценного полезного инструмента в жизни каждого разработчик, позволяющему значительно ускорить процесс разработки и принятий решений относительно такого фундаментального вопроса, как архитектура программного обеспечения.

В заключение автор выражает благодарность и большую признательность научному руководителю
Калашникову~Е.\,А. за поддержку, помощь, обсуждение результатов и~научное
руководство. Автор также благодарит всех, кто сделал настоящую работу автора возможной.
