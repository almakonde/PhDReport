\chapter*{Словарь терминов}             % Заголовок
\addcontentsline{toc}{chapter}{Словарь терминов}\label{ch:tez}  % Добавляем его в оглавление

\textbf{Автоматизированные системы управления или АСУ} : Комплекс средства аппаратного или программного происхождения, предназначенные для выполнения управления технологическими процессами предприятия, производства с участием человека (оператора технологического процесса). Задачей АСУ является повышение эффективности работы технологических процессов за счет повышения производительности труда на предприятии с помощью внедрения средств автоматизации процессов и труда оператора, а также повышение эффективности работы предприятия за счет совершенствования методов планирования процесса управления. Существует несколько видов АСУ на основе объектов управления: автоматизированные системы управления технологическими процессами, автоматизированные системы управления предприятием, автоматизированные систему управления отраслью, фнукциональные автоматизированные системы управления.  

\textbf{Высоконагруженные системы} : Высоконагруженными системами называются системы, которые работают и должны работать под высокими нагрузками. Приложения, которые испытывают высокую популярность при нагрузках: миллионы пользователей одновременно запрашивают доступ к ресурсу.

\textbf{Высоконагруженные данными системы} : Приложение является высоконагруженным данными, в том случае, если приложение интенсивно использует данные, при обработке которых критическими параметрами являются — качество данных, сложность или скорость изменения, — в отличие от приложений, интенсивно использующих вычислительные ресурсы.
где узким местом являются циклы процессора.

\textbf{Надежность} :  Надежность - включает в себя такие понятия, как устойчивость к аппаратным и программным сбоям. 

\textbf{Масштабируемость} : Масштабируемость - включает в себя такие понятия, как показатели нагрузки и производительности; время ожидания, процентили и пропускная способность.

\textbf{Удобство сопровождения} :  Удобство сопровождения - удобство эксплуатации, простота и возможность развития.

\textbf{Проектирование программного обеспечения} : Проектирование программного обеспечения - задача подбора инструментов и инфраструктуры для работы программного обеспечения.

\textbf{Архитектура программного обеспечения} :  Архитектура программного обеспечения - внутренная организация программного обеспечения.

\textbf{Искусственные интеллект} :  Искусственный интеллект - система интеллектуальной модели аппроксимации отношений между объектами и субъектами.

\textbf{Машинное обучение} :  Машинное обучение - это класс методов автоматического создания прогнозных моделей на основе данных. Алгоритмы машинного обучения превращают набор данных в модель. Какой алгоритм работает лучше всего (контролируемый, неконтролируемый, классификация, регрессия и т. д.), зависит от типа решаемой задачи, доступных вычислительных ресурсов и характера данных.

\textbf{Тестовая выборка} :  Тестовая выборка - выборка, на которой обучается модель нейронной сети.

\textbf{Тренировочные данные} :  Тренировочные данные - выборка, на которой проводится проверка результатов работы нейронной сети.

\textbf{Признак} :  Признак – это индивидуальное измеримое свойство или характеристика наблюдаемого явления. Понятие «признак» связано с понятием независимой переменной, которая используется в статистических методах, таких как линейная регрессия. Векторы признаков объединяют все признаки одной строки в числовой вектор.

\textbf{Объект} :  Объектом познания может быть представлена часть реального мира, наблюдаемая, как единое целое в течение длительного времени. Структура объекта может быть: материальной, абстрактной, естественной и искусственной. Объект познания облажает многочисленным набором свойств, в ходе проведения научного исследования разрешается уменьшать размерность свойств для выявления закономерностей и получение исследовательских данных для обработки в соответствии с целями познания и возможностями восприятия. Процесс наблюдения состоит из следующих этапов: назначение переменных, назначение параметров и назначение канала наблюдения.

\textbf{Система } :  Система есть комплекс элементов, находящийся в некотором взаимодействии относительно друг друга. Система — это некоторое множество объектов с множеством отношений между объектами.
Система — это множество, состоящее из элементов, которые находятся в отношениях между собой, связях друг с другом, такое множество объектов образует некоторую обощенную целостность, как объект восприятия или органическое единство. Система — состоит из некоторого целостного набор элементов (компонентов), взаимосвязанных и взаимодействующих между собой некоторым образом, чтобы реализовать некоторое поведение или функционал системы.

\textbf{Общая теория систем (ОТС)} : Общая теория систем (ОТС) — это раздел науки, т.е. научная дисциплина, которая изучает фундаментальные понятия и аспекты работы систем. Она изучает различные явления, отвлекаясь от их конкретной природы и основываясь лишь на формальных взаимосвязях между различными составляющими их факторами и на характере их изменения под влиянием внешних условий, при этом результаты всех наблюдений объясняются лишь взаимодействием их компонентов, например характером их организации и функционирования, а не с помощью непосредственного обращения к природе вовлечённых в явления механизмов (будь они физическими, биологическими, экологическими, социологическими, или концептуальными).

\textbf{Ресурсы вычислительной системы} : Вычислительными ресурсами называются возможности, которые обеспечиваются компонентками вычислительной системы, необходимые для обсуживания вычислительных процессов в системе. 

\textbf{Процесс} : Процессом в информационно-вычислительной системе называют действие, необходимое для сбора, хранения и обработки данных.

\textbf{Состояние} : Под состоянием в информационно-вычислительной системе подразумевается состояние информации в которой находится информационный процесс в системе.

\textbf{Производительность системы} : Количество информационных услуг, предоставляемых за единицу времени.

\textbf{Нагрузка} : Количество транзакций, поступающих на процесс или систему за единицу времени.

\textbf{Новый функционал} : Совокупность новых функциональных возможностей системы.

\textbf{Гарантия доставки сообщений} : Вероятность доставки сообщения за единицу времени.

\textbf{Время обслуживания сообщения} : Оценка времени работы обслуживания процессом обработки сообщения за некоторый промежуток времени.

\textbf{Информационная среда} : Информаионная среда представляет собой совокупность информационных условий существования субъекта (это наличие информационных ресурсов, качество информационных ресурсов, развитие информационной инфраструктуры). Информационная среда представляет условия для развития субъекта информационного пространства, однако, степень ее благоприятствования определяется уже внутренними характеристиками субъекта (информационный потенциал, характеризуемый некоторой априорной информированностью, когнитивностью, определенным уровнем инфопотребности). Условно информационную среду можно разделить на следующие уровни: глобальный – международный и общегосударственный, региональный – субъектный, локальный – городской и сельских местностей.

\textbf{Информационное пространство} :  Пространство, в котором происходит обмен, хранение и обработка информации.

\textbf{САПР} : Системы автоматизированного проектирования/изготовления (Computer Aided Design / Computer Aided Manufacturing - CAD/CAM);

\textbf{АСТПП} : Автоматизированные системы технологической подготовки производства (Computer Aided Engineering - CAE);

\textbf{АСУТП } : АСУТП - автоматизированные системы управления технологическими процессами (Supervisory Control And Data Acquisition - SCADA);

\textbf{АСУП } : АСУП - комплексная автоматизированная система управления предприятием (Enterprise Resource Planning - ERP)

\textbf{WF } : WF - потоки работ (WorkFlow);

\textbf{CRM } : CRM - управление отношениями с клиентами;

\textbf{B2B} : B2B - электронная торговая площадка ("онлайновый бизнес");

\textbf{DSS} : DSS - поддержка принятия управленческих решений;

\textbf{SPSS} : SPSS - статистический анализ данных;

\textbf{OLAP} : OLAP - анализ многомерных данных;

\textbf{MIS} : MIS - управляющая информационная система, (АРМ) руководителя;

\textbf{SCM} : SCM - управление цепями поставок;

\textbf{PLM} : PLM - управление жизненным циклом продукции (характерно для дискретного производства);

\textbf{ERP-II} : ERP-II - расширение ERP-системы за контуры производства (т. е. ERP + CRM + B2B + DSS + SCM+ PLM и т. п.);

\textbf{HR} : HR - "Управление персоналом"; можно рассматривать и как самостоятельную задачу, и как входящую в состав ERP (что и отображено на рисунке в виде двух связей);

\textbf{LAN} : LAN - локальные вычислительные сети (Local Area Net);

\textbf{WAN} : WAN - глобальные (внешние) сети и телекоммуникации (Wide Area Net)

\textbf{Техническое обеспечение АСУ} : Техническое обеспечение АСУ — это комплекс технических средств для выполнения работы системы. КТС состоит из совокупности вычислительных, управляющих устройств, средств преобразования, отображения, регистрации сигналов, устройств передачи сигналов и данных, исполнительных устройств и других элементов, достаточных для выполнения всех функций системы.

\textbf{Программное обеспечение} : Программное обеспечение состоит из совокупности программ, которое разрабатывается для информационного взаимодействия объектов автоматизации для требуемого функционирования информационного взаимодействия систем в соответствии с принятыми математическими методами. Программное обеспечение условно разделяется на два типа: общее и специальное. Общее программное обеспечение ставит своей целью разработку программных средств и инструментов, поставляемых в комплекте со средствами вычислительной техники для общего применения. В отличие от общего программного обеспечения специальное программное обеспечение разрабатывается применительно к конкретной задаче.

\textbf{Информационное обеспечение} : Информационное обеспечение - это совокупность программно-технических средств, направленных на организацию информации, согласно требуемым задачам  АСУ.

\textbf{ Организационное обеспечение} : Организационное обеспечение является совокупностью описаний функциональной, технической и организационной структур, инструкций и регламентов для оперативного персонала на предмет обеспечения заданного режима действий последнего.

\textbf{ Метрологическое обеспечение} : Метрологическое обеспечение включает метрологические средства измерения технологических параметров и инструкции по их применению.

\textbf{ Правовое обеспечение} : Правовым обеспечением являются законодательные акты, регламентирующие взаимоотношения между АСУ и социальной средой.

\textbf{ Лингвистическое обеспечение} : Лингвистическим обеспечением служат языки программирования общего и специального назначения.

\textbf{ Оперативный персонал} : Оперативный персонал АСУТП состоит из технологов-операторов автоматизированного технологического комплекса, осуществляющих контроль технологического процесса и управление ТОУ, и эксплуатационного персонала АСУТП, который обеспечивает заданное функционирование системы в целом.

\textbf{Комплекс технических средств} : комплекс, состоящий из средств технического назначения.

\textbf{Автоматизированная система} : Автоматизированная система — система, состоящая из персонала и комплекса средств автоматизации его деятельности, реализующая информационную технологию выполнения установленных функций.

\textbf{Интегрированная АС} : Интегрированная АС — совокупность двух или более взаимоувязанных АС, в которой функционирование одной из них зависит от результатов функционирования другой (других) так, что эту совокупность можно рассматривать как единую АС.

\textbf{Функция АС} : Функция АС — совокупность действий АС, направленная на достижение определенной цели.

\textbf{Алгоритм функционирования АС} : Алгоритм функционирования АС — алгоритм, задающий условия и последовательность действий компонентов автоматизированной системы при выполнении ею своих функций.

\textbf{ Пользователь АС} : Пользователь АС — лицо, участвующее в функционировании АС или использующее результаты ее функционирования.

\textbf{ Организационное обеспечение АС} : Организационное обеспечение АС — совокупность документов, устанавливающих организационную структуру, права и обязанности пользователей и эксплуатационного персонала АС в условиях функционирования, проверки и обеспечения работоспособности АС.

\textbf{ Методическое обеспечение АС} : Методическое обеспечение АС — совокупность документов, описывающих технологию функционирования АС, методы выбора и применения пользователями технологических приемов для получения конкретных результатов при функционировании АС.

\textbf{ Техническое обеспечение АС} : Техническое обеспечение АС — совокупность всех технических средств, используемых при функционировании АС.

\textbf{ Математическое обеспечение АС} : Математическое обеспечение АС — совокупность математических методов, моделей и алгоритмов, примененных в АС.

\textbf{ Программное обеспечение АС} : Программное обеспечение АС — совокупность программ на носителях данных и программных документов, предназначенная для отладки, функционирования и проверки работоспособности АС.

\textbf{ Информационное обеспечение АС} : Информационное обеспечение АС — совокупность форм документов, классификаторов, нормативной базы и реализованных решений по объемами размещению и формам существования информации, применяемой в АС при ее функционировании.

\textbf{ Лингвистическое обеспечение АС} : Лингвистическое обеспечение АС — совокупность средств и правил для формализации естественного языка, используемых при общении пользователей и эксплуатационного персонала АС с комплексом средств автоматизации при функционировании АС.

\textbf{ Правовое обеспечение АС } : Правовое обеспечение АС — совокупность правовых норм, регламентирующих правовые отношения при функционировании АС и юридический статус результатов ее функционирования.

\textbf{ Эргономическое обеспечение АС} : Эргономическое обеспечение АС — совокупность реализованных в АС решений по согласованию психологических, психофизиологических, антропометрических, физиологических характеристик и возможностей пользователей АС с техническими характеристиками комплекса средств автоматизации АС и параметрами рабочей среды на рабочих местах персонала АС.

\textbf{ Компонент АС} : Компонент АС — часть АС, выделенная по определенному признаку или совокупности признаков и рассматриваемая как единое целое.

\textbf{ Информационная база АС} : Информационная база АС — совокупность упорядоченной информации, используемой при функционировании АС.

\textbf{ Внемашинная информационная база АС} : Внемашинная информационная база АС — часть информационной базы АС, представляющая совокупность документов, предназначенных для непосредственного восприятия человеком без применения средств вычислительной техники.

\textbf{ Машинная информационная база АС} :  Машинная информационная база АС — часть информационной базы АС, представляющая совокупность используемой в АС информации на носителях данных.

\textbf{ Автоматизированное рабочее место (АРМ)} : Автоматизированное рабочее место (АРМ) — программно-технический комплекс АС, предназначенный для автоматизации деятельности определенного вида.

\textbf{ Эффективность АС} : Эффективность АС — Свойство АС, характеризуемое степенью достижения целей, поставленных при ее создании.

\textbf{ Совместимость АС} : Совместимость АС — комплексное свойство двух или более АС, характеризуемое их способностью взаимодействовать при функционировании (включающее техническую, программную, информационную. организационную и лингвистическую совместимость).

\textbf{ Адаптивность АС} : Адаптивность АС — способность АС изменяться для сохранения своих эксплуатационных показателей в заданных пределах при изменениях внешней среды.

\textbf{ Надежность АС} : Надежность АС — комплексное свойство АС сохранять во времени в установленных пределах значения всех параметров, характеризующих способность АС выполнять свои функции в заданных режимах и условиях эксплуатации.

\textbf{ Живучесть АС} : Живучесть АС — свойство АС, характеризуемое способностью выполнять установленный объем функций в условиях воздействий внешней среды и отказов компонентов системы в заданных пределах.

\textbf{ Жизненный цикл АС} : Жизненный цикл АС — совокупность взаимосвязанных процессов создания и последовательного изменения состояния АС от формирования исходных требований к ней до окончания эксплуатации и утилизации комплекса средств автоматизации АС.

\textbf{ Сопровождение АС} : Сопровождение АС — деятельность по оказанию услуг, необходимых для обеспечения устойчивого функционирования или развития АС.

\textbf{ Диалоговый режим выполнения функции АС} : Диалоговый режим выполнения функции АС — режим выполнения функции АС, при котором человек управляет решением задачи, изменяя ее условия и/или порядок функционирования АС на основе оценки информации, представляемой ему техническими средствами АС.

\textbf{ Техническое задание на АС} : Техническое задание на АС — документ, оформленный в установленном порядке и определяющий цели создания АС, требования к АС и основные исходные данные, необходимые для ее разработки, а также план-график создания АС.

\textbf{ Технический проект АС} : Технический проект АС — комплект проектных документов на АС, разрабатываемый на стадии «Технический проект», утвержденный в установленном порядке, содержащий основные проектные решения по системе в целом, ее функциям и всем видам обеспечения АС и достаточный для разработки рабочей документации на АС.

\textbf{ Рабочая документация на АС} : Рабочая документация на АС — комплект проектных документов на АС. разрабатываемый на стадии «Рабочая документация», содержащий взаимоувязанные решения по системе в целом, ее функциям, всем видам обеспечения АС, достаточные для комплектации, монтажа, наладки и функционирования АС, ее проверки и обеспечения работоспособности.

\textbf{ Эксплуатационная документация на АС} : Эксплуатационная документация на АС — часть рабочей документации на АС, предназначенная для использования при функционировании (включающее техническую, программную; информационную, организационную и лингвистическую совместимость}.

\textbf{ Технорабочий проект АС} : Технорабочий проект АС — комплект проектных документов АС, утвержденный в установленном порядке и содержащий решения в объеме технического проекта и рабочей документации на АС.

\textbf{ Входная информация АС } : Входная информация АС — информация, поступающая в АС в виде документов, сообщений, данных, сигналов, необходимая для выполнения функций АС.

\textbf{ Выходная информация АС } : Выходная информация АС — информация, получаемая в результате выполнения функций АС и выдаваемая на объект ее деятельности, пользователю или в другие системы.

\textbf{ Нормативно-справочная информация АС } : Нормативно-справочная информация АС — информация, заимствованная из нормативных документов и справочников и используемая при функционировании АС.

\textbf{ Планирование } : Планирование — это определение цели управления и средств ее достижения, составление плана действий. Оно включает прогнозирование развития и моделирование управляемых объектов.

\textbf{ Организация  } : Организация — это определение круга работ, намеченных планом, определение характера взаимоотношений между производственными и управленческими подразделениями.

\textbf{ Контроль  } : Контроль является аналитической функцией. Он осуществляется непрерывно. В процессе контроля снимаются показания с объекта управления и производится необходимый анализ информации о состоянии процессов производства, выявляются отклонения процессов плана, производится сравнительный анализ.

\textbf{ Регулирование  } : Регулирование — это комплекс операций по устранению возникающих в управляемых системах отклонений.

\textbf{  Учет  } : Учет, в отличие от контрольно-регулирующих функций, имеющих непрерывный характер, дискретен. Он призван концентрировать итоговую информацию, систематизировать ее, создавать по результатам действия объекта управления информационную базу, являющуюся основой при разработке плана действий системы в последующий период.

\textbf{ Управленческий труд } : Управленческий труд — это особая разновидность производительного труда, главной специфической особенностью которого является не непосредственное создание материальных благ, а функциональное управление работниками, производящими эти блага.