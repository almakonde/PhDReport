\chapter*{Словарь терминов}             % Заголовок
\addcontentsline{toc}{chapter}{Словарь терминов}\label{ch:tez}  % Добавляем его в оглавление


\textbf{Определение 1.} : Высоконагруженными системами называются системы, которые работают и должны работать под высокими нагрузками. 

\textbf{Определение 2.} : Приложения, которые испытывают высокую популярность при нагрузках: миллионы пользователей одновременно запрашивают доступ к ресурсу.


\textbf{Определение 3.} : Приложение является высоконагруженным данными, в том случае, если приложение интенсивно использует данные, при обработке которых критическими параметрами являются — качество данных, сложность или скорость изменения, — в отличие от приложений, интенсивно использующих вычислительные ресурсы.
где узким местом являются циклы процессора.

\textbf{Определение 4.} :  Надежность - включает в себя такие понятия, как устойчивость к аппаратным и программным сбоям. 

\textbf{Определение 5.} : Масштабируемость - включает в себя такие понятия, как показатели нагрузки и производительности; время ожидания, процентили и пропускная способность.

\textbf{Определение 6.} :  Удобство сопровождения - удобство эксплуатации, простота и возможность развития.

\textbf{Определение 7.} : Проектирование программного обеспечения - задача подбора инструментов и инфраструктуры для работы программного обеспечения.

\textbf{Определение 8.} :  Архитектура программного обеспечения - внутренная организация программного обеспечения.

\textbf{Определение 9.} :  Искусственный интеллект - система интеллектуальной модели аппроксимации отношений между объектами и субъектами.

\textbf{Определение 10.} :  Машинное обучение - это класс методов автоматического создания прогнозных моделей на основе данных. Алгоритмы машинного обучения превращают набор данных в модель. Какой алгоритм работает лучше всего (контролируемый, неконтролируемый, классификация, регрессия и т. д.), зависит от типа решаемой задачи, доступных вычислительных ресурсов и характера данных.

\textbf{Определение 11.} :  Тестовая выборка - выборка, на которой обучается модель нейронной сети.

\textbf{Определение 12.} :  Тренировочные данные - выборка, на которой проводится проверка результатов работы нейронной сети.

\textbf{Определение 13.} :  Признак – это индивидуальное измеримое свойство или характеристика наблюдаемого явления. Понятие «признак» связано с понятием независимой переменной, которая используется в статистических методах, таких как линейная регрессия. Векторы признаков объединяют все признаки одной строки в числовой вектор.

\textbf{Определение 14.} :  Объектом познания может быть представлена часть реального мира, наблюдаемая, как единое целое в течение длительного времени. Структура объекта может быть: материальной, абстрактной, естественной и искусственной. Объект познания облажает многочисленным набором свойств, в ходе проведения научного исследования разрешается уменьшать размерность свойств для выявления закономерностей и получение исследовательских данных для обработки в соответствии с целями познания и возможностями восприятия. Процесс наблюдения состоит из следующих этапов: назначение переменных, назначение параметров и назначение канала наблюдения.

\textbf{Определение 15.} :  Система есть комплекс элементов, находящийся в некотором взаимодействии относительно друг друга. Система — это некоторое множество объектов с множеством отношений между объектами.
Система — это множество, состоящее из элементов, которые находятся в отношениях между собой, связях друг с другом, такое множество объектов образует некоторую обощенную целостность, как объект восприятия или органическое единство. Система — состоит из некоторого целостного набор элементов (компонентов), взаимосвязанных и взаимодействующих между собой некоторым образом, чтобы реализовать некоторое поведение или функционал системы.

\textbf{Определение 16.} : Общая теория систем (ОТС) — это раздел науки, т.е. научная дисциплина, которая изучает фундаментальные понятия и аспекты работы систем. Она изучает различные явления, отвлекаясь от их конкретной природы и основываясь лишь на формальных взаимосвязях между различными составляющими их факторами и на характере их изменения под влиянием внешних условий, при этом результаты всех наблюдений объясняются лишь взаимодействием их компонентов, например характером их организации и функционирования, а не с помощью непосредственного обращения к природе вовлечённых в явления механизмов (будь они физическими, биологическими, экологическими, социологическими, или концептуальными).
