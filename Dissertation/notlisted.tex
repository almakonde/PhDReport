
\subsection{Математическая модель системы логического вывода}\label{sec:ch2/sec3/sub1}

\subsection{Алгоритмическое решение проблемы проектирования устойчивой архитектуры программного обеспечения}\label{sec:ch2/sec3/sub2}
\section{Решение проблемы масштабирования архитектуры программного обеспечения автоматизированной системы управления}\label{sec:ch2/sec4}
\subsection{Математическая модель масштабирования архитектуры программного обеспечения}\label{sec:ch2/sec4/sub1}
\subsection{Алгоритмическое решение проблемы масштабирования программного обеспечения высоконагруженной системы}\label{sec:ch2/sec4/sub2}
\section{Решение проблемы выбора архитектуры программного обеспечения автоматизированной системы управления}\label{sec:ch2/sec5}
\subsection{Математическая модель выбора архитектуры программного обеспечения автоматизированной системы управления}\label{sec:ch2/sec5/sub1}
\subsection{Алгоритмическое решение выбора архитектуры программного обеспечения автоматизированной системы управления}\label{sec:ch2/sec5/sub2}
\section{Решение проблемы обновления программного обеспечения автоматизированной системы управления}\label{sec:ch2/sec6}
\subsection{Математическая модель обновления программного обеспечения автоматизированной системы управления}\label{sec:ch2/sec6/sub1}
\subsection{Алгоритмическое решение обновления программного обеспечения автоматизированной системы управления}\label{sec:ch2/sec6/sub2}
\section{Решение проблемы обеспечения доступности, целостности, конфиденциальности информации при проектировании архитектуры программного обеспечния автоматизированной системы управления}\label{sec:ch2/sec7}
\subsection{Математическая модель обеспечения доступности, целостности, конфиденциальности информации при проектировании архитектуры программного обеспечния автоматизированной системы управления}\label{sec:ch2/sec7/sub1}
\subsection{Алгоритмическое решение обеспечения доступности, целостности, конфиденциальности информации при проектировании архитектуры программного обеспечния автоматизированной системы управления}\label{sec:ch2/sec7/sub2}
\chapter{Методика оценки надежности специального математического и алгоритмического обеспечения проектирования архитектуры высоконагруженной информационной системы}\label{ch:ch5}

\section{Математические модели оценки показателей качества специального математического и алгоритмического обеспечения проектирования архитектуры высоконагруженной информационной системы}\label{sec:ch3/sect1}
\subsection{Модель полноты выполнения функций}\label{subsec:ch3/sect2/sub1}
\subsection{Модель своевременности выполнения функций}\label{subsec:ch3/sect2/sub2}
\subsection{Показатели достоверности функционирования}\label{subsec:ch3/sect2/sub3}
\subsection{Показатели надежности функционирования}\label{subsec:ch3/sect2/sub4}
\subsection{Понятие отказа и сбоя в модели надежности}\label{subsec:ch3/sect2/sub5}
\subsection{Схема надежности}\label{subsec:ch3/sect2/sub6}

\section{Методика оценки надежности специального математического и алгоритмического обеспечения проектирования архитектуры высоконагруженной информационной системы}\label{sec:ch3/sect3}
\subsection{Математический аппарат оценки надежности технического обеспечения специального математического и алгоритмического обеспечения проектирования архитектуры высоконагруженной информационной системы}\label{subsec:ch3/sect3/sub1}
\subsection{Оценка показателей надежности в условиях неполных статистических данных}\label{sec:ch3/sect3/sub2}
\subsection{Оценка границ показателей надежности}\label{sec:ch3/sect3/sub3}
\section{Порядок работы по методике}\label{sec:ch3/sect4}
\subsection{Порядок проведения экспресс оценки надежности}\label{sec:ch3/sect4/sub1}
\subsection{Порядок получения уточненной оценки надежности}\label{sec:ch3/sect4/sub2}
\section{Методика автоматизированного сбора статистических данных}\label{sec:ch3/sect5}
\subsection{Общие принципы организации сбора статистических данных по отказам и сбоям}\label{sec:ch3/sect5/sub1}
\subsection{Порядок сбора }\label{sec:ch3/sect5/sub2}
\subsection{Оценка показателей надежности в условиях неполных статистических данных}\label{sec:ch3/sect5/sub3}

\section{Выводы по главе}\label{sec:ch3/conc}

\clearpage

\chapter{Алгоритм оценки устойчивости специального математического и алгоритмического обеспечения проектирования архитектуры высоконагруженной информационной системы}\label{ch:ch6}

\section{Основные понятия и определения}\label{sec:ch6/sect1}
\section{Проблемы обеспечения устойчивости к внешним воздействиям и пути их решения}\label{sec:ch6/sect2}

\subsection{Общая характеристика проблем обеспечения устойчивости к внешним условиям}\label{subsec:ch6/sect2/sub1}
\subsection{Основные положения алгоритмического аппарата оценки устойчивости к внешним воздействиям}\label{subsec:ch6/sect2/sub2}
\subsection{Ограничения и допущения модели оценки устойчивости}\label{subsec:ch6/sect2/sub3}

\section{Алгоритм оценки устойчивости к внешним воздействиям}\label{sec:ch6/sect3}
\section{Методология оценки влияния человеческого фактора на функционирование специального математического и алгоритмического обеспечения проектирования архитектуры высоконагруженной информационной системы}\label{sec:ch6/sect4}
\subsection{Основные понятия и определения}\label{subsec:ch6/sect4/sub1}
\subsection{Математическая модель оценки влияния человеческого фактора на безошибочное функционирование}\label{subsec:ch6/sect4/sub1}
\subsection{Алгоритм оценки влияния человеческого фактора на устойчивость функицонирования}\label{subsec:ch6/sect4/sub1}
\subsection{Методологический подход к оценке влияния человеческого фактора на устойчивость функионирвования при отсутствии точных статистических данных}\label{subsec:ch6/sect4/sub1}
\section{Выводы по главе}\label{sec:ch6/sect3}


\clearpage
