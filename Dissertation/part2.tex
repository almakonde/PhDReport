\chapter{Разработка моделей и алгоритмов обеспечения проектирования архитектуры автоматизированной системы управления}\label{ch:ch2}
\section{Типы автоматизированных систем управления}\label{sec:ch2/sec1}
\subsection{Типы транзакционных систем}\label{sec:ch2/sec1/sub1}
Информационные потоки в транзакционных системах чаще всего классифицируются по типу хранения и обработки информационного потока данных. Классифицируем данные по следующим признакам:
\begin{enumerate}
	\item данные по деталям, чаще всего обработка элементарных данных,
	\item агрегированные данные, чаще всего суммирование по некоторым измерениям,
	\item метаданные, т.е. данные, которые могут описывать некоторые конфигурационные действия над данными (обстоятельства использвания данных).
\end{enumerate}
Данные образуют некоторый инфомационный поток, который может быть разделен по следующим направлениям:
\begin{enumerate}
	\item входной поток,получаемый системой при обработке (источником информации может служить любой объект),
	\item потоки сообщений, которые образуются в результате работы деловой логики взаимодействия объектов,
	\item архивные потоки, образуются из классов данных, спрос на выполнение которых стал устаревать,
	\item поток метаданных, данные, которые образовались за счет конфигурационных, логгированных, журналированных даннных,
	\item выходной поток, данные, которые получает пользователь на выходе из работы системы,
	\item  обратный поток, данные, которые возвращаются в систему.
\end{enumerate}
\subsection{OLTP системы}\label{sec:ch2/sec1/sub2}
OLTP  транзакционная система - система обработки транзакций в реальном времени. Такие системы предназначены для ввода, обработки и структурирования информации в режиме реального времени. К таким системам обычно предъявляются следующие требования:
\begin{enumerate}
 	\item ограничения на обработку информации через реальное время, 
 	\item транзакционные модели данных должны быть сильно нормализованы,
 	\item в случае возникновения ошибок, транзакции должны возвращаться в исходное состояние, до момента начала совершения транзакции,
 \end{enumerate}
Компоненты исполнения транзакций являются частью компонентов нагрузки, исполнение которых ограничиваются строгими требованиями выполнения в определенный интервал времени. Более того, транзакции, после отработки постоянно обновляют базу данных. Для проектировании подобных систем следует учитывать следующие параметры и характеристики работы программного обеспечения:
\begin{enumerate}
	\item при проведении транзакции используется относительно небольшой объем данных,
	\item индексы в базу данных должны быть легко доступны для уменьшения времени обработки транзакции,
	\item система должна иметь относительно небольшой промежуток временного отклика,
	\item используются только нормализованные формы транзакционные модели,
	\item может поддерживать сложные транзакционные модели,
	\item модель обработки транзакций проектируется опираясь на требования по нижнему пределу ограничения пропускной способности информации.
\end{enumerate}
Таким образом, основная цель OLTP систем - предоставление доступа в ограниченном отрезке реального времени, в строгом соответствии с нижним пределом пропускной способности. Суть таких систем состоит в соблюдении ограничений по обработке транзакций в строго определенном режиме обработки: ограниченного во времени, ограниченного в нижнем пределе по нагрузке. Система должна обеспечивать вычислительную нагрузку в реальном времени в стогом соответствии с ограничениями по времени и пропускной способности. Система должна восстанавливаться  в исходное состояние транзакций в случае выявления ошибок обработки. Система должна обновлять значения базы данных после проведения каждой успешной транзакции. В системе основным критерием успешной работы является обработка транзакции за ограниченный период в реальном времени. Такие системы не ориентированы на сохранение данных, при надобности данные системы архивируются.
\subsection{OLAP системы}\label{sec:ch2/sec1/sub3}
OLAP системы - это системы, которые ориентированы на аналитическую обработку данных. Такие системы анализируют данные многомерные данные и чаще всего используются для построения анализа многомерных систем.
Цель таких систем, предоставление аналитических отчетов, поэтому фундаментом обработки данных становится база данных, основанная на фактах. Таблица фактов - является сердцем таких систем, правильно спроектированная таблица фактов, которых соблюдаются все правильно низкой связности и высокой группировки, обеспечивает быстрое время обработки запросов в таких системах. 
Таблица фактов содержит фактическую информацию о свойстве объектов или событий. Факты делятся по следующим признакам:
\begin{enumerate}
	\item факты о транзакциях, т.е. факты, хранящие информацию о конретных совершенных действиях, 
	\item факты о состояниях,
	\item факты о свойствах того или иного объекта,
	\item факты о событиях.
\end{enumerate}
Таблица фактов должна содержать один внешний составной ключ, который должен объединять первичные ключи таблиц измерений. В настощее время разработано множество стандартов для проектирования OLAP систем. В общем случае, архитектура таких систем относится к обработке анализа, хранения и преобразования данных в независимых слоях системы, абстрактно ограниченных программным кодом. Основная нагрузка в таких системах ложится на организацию составления аналитических данных, где могут применяться различными методы решения вывода и аггрегации инфомации: например, путем изоляции обращений к базе данных, через объектно реляционные запросы на высокоуровневом языке программирования, либо оптимизация запросов к базе данных. 
Подходы к проектированию систем будут рассмотрены более подробно далее в работе.

\subsection{Типы предоставления доступа}\label{sec:ch2/sec1/sub1}
Системы автоматизированного управления могут быть разделены по следующим моделям предоставления доступа:
\begin{enumerate}
	\item SaaS модель предоставления доступа,
	\item PaaS модель предоставления доступа,
	\item IaaS модель предоставления доступа.
\end{enumerate}

\subsection{SaaS системы}\label{sec:ch2/sec1/sub4} 

SaaS система - система организации кода в виде предоставления доступа к программному обеспечению, как к услуги. 
Программное обеспечение системы SaaS обладает следующими ключевыми признаками:
\begin{enumerate}
    \item доступ к программному обеспечению, разработанному в соответствии с моделью программного обеспечения, как услуга, предоставляется удалённо по сетевым каналам и, как правило, через веб-интерфейс, кроме того, могут использоваться тонкие клиенты и терминальный доступ;
    \item программное обеспечение развёртывается в центре обработки данных в виде единого программного ядра, с которым работают все заказчики;
    \item обслуживание и обновление программного обеспечения выполняется централизованно на стороне поставщика приложения, предоставляемого как услуга (SaaS).
\end{enumerate}

\subsection{PaaS системы}\label{sec:ch2/sec1/sub5}
PaaS система предоставляет доступ к программному обеспечению через платформу. Платформа, предоставляется, как услуга. Чаще всего модель предоставляется, как облачный сервер с набором реализованного функционала или инструментария в аренду под конкретные цели. Модель, предоставляется за счет облачных вычислений, реализованных на облачном сервере. Функционал информационно-вычислительной инфраструктуры, включая вычислительные серверы, серверы, системы хранения, целиком управляется провайдером услуг. 


\subsection{IaaS системы}\label{sec:ch2/sec1/sub6}
IaaS системы - это такой тип систем, который предоставляет инфроструктуру как сервис. Чаще всего данные размещаются на облаке или в гибридном доступе, функционал по разработке инфраструктуры предоставляется по подписке. С помощью модели IaaS инфраструктура располагается на облаке, как набор сервисов.
С помощью модели IaaS компании могут частично или полностью переместить в облако локальную инфраструктуру центра обработки данных, где ее обслуживанием и управлением занимается поставщик облачных сервисов. К числу таких экономически эффективных элементов инфраструктуры могут относиться вычислительные и сетевые ресурсы, оборудование для хранения данных, а также другие компоненты и программное обеспечение. 
В рамках стандартной модели IaaS компании любого размера используют различные сервисы, такие как вычислительные ресурсы, хранилище и базы данных, предоставляемые поставщиком облачных решений. Поставщик услуг предоставляет эти сервисы путем размещения оборудования и программного обеспечения в облаке. Компании при этом не требуется приобретать собственное оборудование, заниматься его администрированием и отводить под него место в своих центрах обработки данных. А затраты она несет по модели «оплата по мере использования». Если компании нужно меньше ресурсов, общие затраты на них снижаются. А по мере роста компания может за считаные минуты предоставить сотрудникам дополнительные вычислительные ресурсы и другие технологии.
\section{Постановка задачи проектирования архитектуры автоматизированных систем управления, основные ограничения и допущения}\label{sec:ch2/sec2}
Архитектура программного обеспечения автоматизированной информационной системы представляет собой взаимосвязанную систему компонентов и модулей, которые стремятся обеспечить решение некоторой поставленной задачи автоматизации. Архитектура состоит из множества технических решений, которые можно комбинировать некоторым способом, архитектура допускает множество технических реализаций путем выбора различных компонентов архитектуры и методов взаимодействия между ними. В данной работе представлена модель архитектуры т.е. модель Библиотекаря для проектирования высоконагруженных автоматизированных информационных систем, с выполнением необходимых свойств, описанных выше.  
Перечислим типовой перечень сервисов и служб, необходимых для проектирования автоматизированной информационной системы:
\begin{enumerate}
	\item организация хранилища данных;
	\item организация обработки данных;
	\item формирование деловых функций объекта автоматизации;
	\item создание пользовательского интерфейса;
	\item разработка каналов обмена и передачи информации для интеграции со сторонними сервисами или ресурсами.
\end{enumerate}
Формализуем нашу задачу. Пусть множество типов информационных систем, описываемых пользователем как  $U=\{u1,u2,...,un\}$, множество объектов, т.е. предлагаемых архитектур $P=\{p1,p2,...,pm\}$, весовая матрица  $R=ri,j$ размера $nxm$,  $i\{1...n\},j\{1...m\}$. $N$ - желаемое количество рекомендаций, которые нужно получить от системы. Набор типов информационных систем состоит из объекта, описываемого как набор некоторых характеристик, которые должны быть заполнены пользователем. Набор характеристик состоит из следующих обязательных пунктов для заполнения с множеством возможных решений, которые необходимо принять:
\begin{enumerate}
\item количество транзакций в секунду на чтение,
\item количество транзакций в секунду на запись,
\item количество пользователей,
\item требования к дальнейшей масштабируемости,
\item функциональные требования,
\item типы объектов в системе.
\end{enumerate}
Требуется найти: для описываемого набора типов информационных систем $u$, найти $N $- мерный вектор $p_{i1},p_{i2},...,p_{iN}$, где архитектура $p_{ik},k\in N$  еще не оценены экспертами, т.е. в матрице описания транзакций есть пустое место $r_i,i_k$, где эти архитектуры наиболее точно соответствуют базе продукционных правил и знаний, то есть прогнозным рейтингам, здесь $r_i,i_k$ является наибольшим. Алгоритмы, необходимые для решения этой задачи, могут быть самыми разными и использовать разные входные данные. Некоторые из них генерируют рекомендации только на основе данных об известных матрицах описания транзакций или на заранее описанных известных продукционных правилах. Другие используют дополнительные характеристики, используют матрицы описания транзакций, чтобы определить, какие из этих характеристик наиболее точно соответствуют предпочтениям пользователя, а затем выбирают альтернативы с этими характеристиками.

\section{Решение проблемы проектирования устойчивой архитектуры программного обеспечения автоматизированной системы управления}\label{sec:ch2/sec3}

Рассмотрим параметрический критерий, основанный на следующем свойстве экспоненциального распределения плотности 
\begin{equation}
    \label{eq:equation20}
    f(t) = \frac{1}{T}e^{(-1T)}, T = \sqrt{D}  
\end{equation}

где $T$ - математическое ожидание экспоненциального закона;

$D$ - дисперсия, определяемая по зависимости ~\cref{eq:equation21}
\begin{equation}
    \label{eq:equation21}
    D = \int_0^{\infty} \mathrm \{\frac{(1-T)^2}{T}e^{-\frac{1}{T}}\,\mathrm{d}t    
\end{equation}

Рассмотрим случай, когда на систему действует два источника нагрузки: пользовательская нагрузка, состоящая из поступающего на обслуживание через веб-сервер множество запросов и запросы командной нагрузки от сторонних платформ, поступающие через интеграционные шлюзы. Запросы проходят через доменные имена и распределяются в балансировщиках нагрузки, которые направляют входящие запросы на один из множества серверов приложения, которые обычно являются зеркальными копиями друг друга, и отправляют ответ обратно пользователю. Любой сервер обрабатывает запросы одинаково, так что балансировщик занимается распределением заданий, чтобы никакой из них не был перегружен.
Таким образом, в момент старта работы приложений запросы между пользователями могут соревноваться за ресурсы информационной системы. Для описания этого процесса воспользуемся двух-параллельным марковским процессом ~\cref{eq:equation22}
\begin{equation}
    \label{eq:equation22}
    M = [A, h(t)]  
\end{equation}
                       (3)
где $ A = \{a_{w1},a_{w2}, a_{g1},a_{g2} \}$ - множество состояний, 

$a_{w1}, a_{g1}$ - стартовые состояния, 

$a_{w2}, a_{g2}$ - поглощающие состояния, 

$h(t)$ - полумарковская матрица;

Для определения времени ожидания процесса воспользуемся описанием вида ~\cref{eq:equation23}


\begin{equation}
    \label{eq:equation23}
    M = [A', h'(t)]  
\end{equation}

где $A' = AB$ - множество состояний;

$A = \{\alpha_1, \alpha_2, \alpha_3\}$ - подмножество состояний, моделирующее начало и окончания блужданий по полумарковскому процессу; 

$\alpha_1$ - стартовое состояние;

$\alpha_2$ - поглощающее состояние, моделирующее выигрыш второго субъекта; 

$\alpha_3$ - поглощающее состояние, моделирующее окончание ожидания первым субъектом окончания второго;

$B = \{\beta_1,\beta_2, \beta_3 \}$ - бесконечное множество состояний, задающих временные интервалы ситуаций завершения обслуживания вторым
 $h'(t) = \{h'_{m,n}(t)\}$ - полумарковская матрица, задающая временные интервалы процесса.


Представим плотность распределения времени наблюдения за системой посредством закона вырожденного распределения с некоторым математическим ожиданием  $T$   и  $\omega(t) = \delta(t - T)$ , который соответствует некоторому детерминированному процессу. Таким образом, плотность распределения времени ожидания завершения события $g(t)$, определяется согласно зависимости ~\cref{eq:equation24}:


\begin{equation}
    \label{eq:equation24}
    f_{\delta \rightarrow g(t)} = \frac{\eta(t)g(t+T)}{\int_T^{\infty} \mathrm {g(t)}}\,\mathrm{d}t}
\end{equation}

Математическое ожидание имеет вид ~\cref{eq:equation25}:
\begin{equation}
    \label{eq:equation25}
    T_{\delta \rightarrow g(t)} = \int_0^{\infty} \mathrm {t\frac{g(t+T)}{\int_T^{\infth} \mathrm{g(t)\,\mathrm{d}t}}}\,\mathrm{d}t}
\end{equation}

Критерий, основанный на определении времени ожидания для строго детерминированной связи между событиями, выражаемой - функции Дирака $g(t) = \delta(t - T)$, имеет вид:
\begin{equation}
    \label{eq:equation26}
    \varepsilon_{\omega} = \frac{t-T_{\delta\rightarrow\g}}{T}^2 
\end{equation}

где $T$ - математическое ожидание анализируемой плотности распределения времени между соседними событиями; 

$T_{\delta\rightarrow g}$- математическое ожидание плотности распределения $f_{\delta\rightarrow g(t)}, рассчитываемое по зависимости ~\cref{eq:equation24}.

Математическое ожидание распределения сигнала определяется по следующей зависимости 
\begin{equation}
    \label{eq:equation27}
    T_k = \int_0^1 \mathrm{tK(1-t)^{K-1}}\,\mathrm{d}t = \frac{1}{K+1}[время]
\end{equation}

Экспоненциальный закон, определяющий Пуассоновский поток событий 
\begin{equation}
    \label{eq:equation28}
    f_K(t) = (K+1)e^{[-(K+1)t]} [probtime]
\end{equation}

\section{Выводы по главе}\label{sec:ch2/conc}

В данной главе решены следующие задачи:
\begin{enumerate}
	\item проведен анализ конкурентных решений на рынке информационных технологий,
	\item систематизированы типы транзакционных моделей, 
	\item продемонстрирована постановка задачи проектирования архитектуры программного обеспечения, рассмотрены основные ограничения, допущения,
	\item продемонстрировано решение проблемы проектирования устойчивой архитетектуры автоматизированной системы управления,
	\item систематизирована информация по наиболее актуальным вопросам транзакционных проблем при проектировании архитектуры программного обеспечения автоматизированных систем управления.
\end{enumerate}
\clearpage