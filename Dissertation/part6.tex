\chapter{Алгоритм оценки устойчивости специального математического и алгоритмического обеспечения проектирования архитектуры высоконагруженной информационной системы}\label{ch:ch6}

\section{Основные понятия и определения}\label{sec:ch6/sect1}
\section{Проблемы обеспечения устойчивости к внешним воздействиям и пути их решения}\label{sec:ch6/sect2}

\subsection{Общая характеристика проблем обеспечения устойчивости к внешним условиям}\label{subsec:ch6/sect2/sub1}
\subsection{Основные положения алгоритмического аппарата оценки устойчивости к внешним воздействиям}\label{subsec:ch6/sect2/sub2}
\subsection{Ограничения и допущения модели оценки устойчивости}\label{subsec:ch6/sect2/sub3}

\section{Алгоритм оценки устойчивости к внешним воздействиям}\label{sec:ch6/sect3}
\section{Методология оценки влияния человеческого фактора на функционирование специального математического и алгоритмического обеспечения проектирования архитектуры высоконагруженной информационной системы}\label{sec:ch6/sect4}
\subsection{Основные понятия и определения}\label{subsec:ch6/sect4/sub1}
\subsection{Математическая модель оценки влияния человеческого фактора на безошибочное функционирование}\label{subsec:ch6/sect4/sub1}
\subsection{Алгоритм оценки влияния человеческого фактора на устойчивость функицонирования}\label{subsec:ch6/sect4/sub1}
\subsection{Методологический подход к оценке влияния человеческого фактора на устойчивость функионирвования при отсутствии точных статистических данных}\label{subsec:ch6/sect4/sub1}
\section{Выводы по главе}\label{sec:ch6/sect3}


\clearpage