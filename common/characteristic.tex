
{\actuality} 

На сегодняшний день модели и алгоритмы проектирования высоконагруженных облачных информационных и вычислительных средств является актуальной т.к. до сих пор не сформировано единых мировых утвержденных стандартов, применяемых в тех или иных видах, типах, классах информационных систем. 
Наряду с тем, что количество вычислительных устройств растет, растет и необходимость в разработке программного обеспечения, возникает необходимость в разработке моделей проектирования архитектуры, методов и средств программирования и создания ВОИВС ориентированных на достижение высоких показателей по следующим характеристикам: устойчивость, надежность, производительность, масштабируемость, адекватность и точность, гибкость, удобство, доступность.
Существует множество подходов при проектировании систем в данной работе ставится цель проанализировать существующие, исследовать и разработать новую модель проектирования. 
Основные результаты диссертации были получены в процессе выполнения работ по следующим направлениям:
\begin{enumerate}
    \item публикация статей в периодических изданиях~\cite{aaij2022searches, baptista2021angular, leite2021observation,baptista2021searches,collaboration2021measurement,baptista2021measurement,aaij2021precision,aaij2022identification,bediaga2020measurement, aaij2021constraints,   onderwater2020study, aaij2022arxiv,     aaij2019arxiv,aaij2021evidence,aaij2020aps,baptista2021observation,baptista2021search,aaij2020isospin,  aaij2019precision,aaij2019search, lhcb2108evidence,aaij2022study,         aaij2022first,aaij2022j,  aaij2022observation, aaij2022tests}.
    \item выступление на конференциях,
    \item технологическое внедрение на промышленном проекте “Лукойл” информационная система обработки тендеров,
    \item технологическое внедрение на ГИИС ДМДК
    \item технологическое внедрение при проектировании медицинского оборудования для офтольмологической диагностики.
\end{enumerate}

Объектом исследования являются автоматизированные системы управления, облачные системы хранения, обработки и управления данными.

Предметом исследования является разработка модели проектирования архитектуры  автоматизированных систем управления, основанных на расчете нагрузки.

Целью данной работы является разработка моделей и алгоритмов проектирования программного обеспечения, разработка моделей для обработки и хранения данных в высоконагруженных, высоковычислительных и стандартных  автоматизированных систем управления.

Для достижения поставленной цели необходимо решить следующие задачи:
\begin{enumerate}
\item Исследовать типы информационных систем;
\item Исследовать типы вычислительных систем;
\item Исследовать модели серверного и облачного хранения данных;
\item Исследовать модели виртуализации данных;
\item Исследовать модели организации доступа к данным в информационных и вычислительных системах;
\item Исследовать модели организации базы данных в информационных и вычислительных системах;
\item Исследовать модели обработки данных в информационных и вычислительных системах;
\item Исследовать модели организации защиты данных, в том числе авторизации и аутентификации пользователей;
\item Определить целевые параметры модели;
\item Определить функционал достижения цели;
\item Рассмотреть задачу используя методологии системного анализа;
\item Определить методику исследования, ограничения;
\item Построить математическую модель
\item Разработать и реализовать алгоритмы, исследуемые методы;
\item Исследовать показатели результативности метода по отношению к целевым параметрам
\end{enumerate}

Теоретическая и методологическая основа исследования - нейронная сеть на основе нечеткого логического вывода.

Методология и методы исследования основаны на теории систем, системном анализе, теории управления, математическом моделировании, нелинейной динамике.

Основные положения, выносимые на защиту:
\begin{enumerate}
\item модель поддержки принятия решений, автоматизирущая процесс разработки программного обеспечения,
\item модель работы автоматизированной системаы проектирования архитектуры программного обеспечения автоматизированных систем управления,
\end{enumerate}



Научная новизна: состоит в методологии проектирования, которая использует базы знаний наряду с алгоритмами машинного обучения для построения модели проектирования архитектуры, что является функционалом, который приносит полезный результат.

Практическая значимость: состоит в том, что полезным результатом являестя возможность ускорить темпы разработки, а также помочь неопытным специалистам спроектировать архитектуру программного обеспечения для понимания полноты системы.

Достоверность: оценивается на основе критериев робстности модели машинного обучения, а также на основе критериев устойчивости.

Апробация работы: прошла аппробацию в двух проектах (справки о внедрении в Приложении).

Личный вклад. Результаты, изложенные в диссертации, принадлежат лично автору. В совместных работах автор принимал непосредственное участие в выборе направлений и задач исследований, разработке и обсуждении результатов.

Публикации: все публикации автора перечислены в~\cite{aaij2022searches, baptista2021angular, leite2021observation,baptista2021searches,collaboration2021measurement,baptista2021measurement,aaij2021precision,aaij2022identification,bediaga2020measurement, aaij2021constraints,   onderwater2020study, aaij2022arxiv,     aaij2019arxiv,aaij2021evidence,aaij2020aps,baptista2021observation,baptista2021search,aaij2020isospin,  aaij2019precision,aaij2019search, lhcb2108evidence,aaij2022study,         aaij2022first,aaij2022j,  aaij2022observation, aaij2022tests}.

Объем и структура работы. Диссертация состоит из введения, 4 глав и заключения. Полный объем диссертации составляет 104 страницы, включая 23 рисунка, 4 таблиц. Список литературы содержит 99 наименований.

Информационная база исследования: модели и принципы построения архитектуры программного обеспечения из открытых источников.

Обоснованность и достоверность результатов исследования: диссертация обоснована на математических представлениях, достоверность закреплена снимками работы программного обеспечения.

Соответствие диссертации Паспорту научной специальности: соответствует.

Научная новизна результатов исследования: разработана нвая модель поддержки принятия решений для проектирования систем автоматизированного управления.

Наиболее существенные результаты исследования, обладающие научной новизной и полученные лично соискателем.

Теоретическая и практическая значимость исследования: разработанная модель поддержки принятий решений несет полезный результат в виде ускорение временных затрат на поиск и анализ необходимой модели архитектуры программного обеспечения автоматизированной системы управления.

Апробация результатов исследования: внедрена при разработке 2 проектов. 

Публикации результатов исследования приведены в~\cite{aaij2022searches, baptista2021angular, leite2021observation,baptista2021searches,collaboration2021measurement,baptista2021measurement,aaij2021precision,aaij2022identification,bediaga2020measurement, aaij2021constraints,   onderwater2020study, aaij2022arxiv,     aaij2019arxiv,aaij2021evidence,aaij2020aps,baptista2021observation,baptista2021search,aaij2020isospin,  aaij2019precision,aaij2019search, lhcb2108evidence,aaij2022study,         aaij2022first,aaij2022j,  aaij2022observation, aaij2022tests}.


Структура диссертации
\begin{enumerate}
\item В первой главе описываются проблемы проектирования автоматизированных систем управления, методолгии и средств, доступные на сегодняшний день.
\item Во второй главе приводится анализ задачи проектирования системы поддержки принятия решений, приводится постановка задачи, анализируются подход в разработке к типовым транзакционным моделям проектирования автоматизированных систем управления и организации данных.
\item В третьей главе представлена  модель специального математического и алгоритмического обеспечения системы анализа, принятия решений и обработки информации в проектировании архитектуры
автоматизированных систем управления.
\item В четвертой главе рассмотрены принципы создания инструмента проектирования архитектуры автоматизированных систем управления, систем поддержки принятия решений на основе нейронных сетей с продукционным нечетким выводом.
\end{enumerate}



\ifsynopsis
Объектом исследования являются информационные системы хранения, обработки и управления данными.

Предметом исследования является разработка модели автоматизации проектирования архитектуры автоматизированных систем, основанных на расчете нагрузки.

Целью данной работы является разработка моделей и алгоритмов проектирования программного обеспечения, разработка моделей для обработки и хранения данных в высоконагруженных, высоковычислительных и стандартных автоматизированных системах.
\else


\fi

% {\progress}
% Этот раздел должен быть отдельным структурным элементом по
% ГОСТ, но он, как правило, включается в описание актуальности
% темы. Нужен он отдельным структурынм элемементом или нет ---
% смотрите другие диссертации вашего совета, скорее всего не нужен.

{\aim} данной работы является  разработка моделей и алгоритмов проектирования программного обеспечения, разработка моделей для обработки и хранения данных в высоконагруженных, высоковычислительных и стандартных автоматизированных системах.

Для~достижения поставленной цели необходимо было решить следующие {\tasks}:
\begin{enumerate}[beginpenalty=10000] % https://tex.stackexchange.com/a/476052/104425
  \item Исследовать модели принятия решений, разработать модуль интеллектуального продукционного вывода, вычислить параметрические отображения входных и выходных переменных,
  \item Исследовать типовые архитектуры проектирования автоматизированных систем управления, разработать универсальный базис оценки критериев, вычислить эвристический механихм продукционного вывода знаний.
\end{enumerate}


{\novelty}
\begin{enumerate}[beginpenalty=10000] % https://tex.stackexchange.com/a/476052/104425
  \item Впервые предложена модель поддержки принятия решений для проектирования архитектуры программного обеспечения автоматизированных систем управления,
  \item Впервые предложена методология универсального проектирования, анализа потребностей с помощью нейронной сети на основе нечеткого вывода для автоматизации процесса создания базы знаний,
  \item Было выполнено оригинальное исследование возможности разработки автоматизированной системы поддержки принятия решений для проектирования архитектуры программного обеспечения автоматизированных системы управления.
\end{enumerate}



{\defpositions}
\begin{enumerate}[beginpenalty=10000] % https://tex.stackexchange.com/a/476052/104425
  \item Первое положение опирается на принципе сбора фактов и систематизации в виде таблицы фактов,
  \item Второе положение опирается на интерпретацию и эвристическое формирование модели отношений между объектами, характеристиками и субъектами,
  \item Третье положение опирается на формирование отображения и поправки модели обучения с помощью возмущающего режима получения ошибки и поправки градиента обучения,
  \item Четвертое положение опирается на механизм соонесения семантических термов относительно классов принадлежности на основе теории нечеткого продукционного вывода и нейронного обучения.
\end{enumerate}


{\probation}
Основные результаты работы докладывались в~\cite{aaij2022searches, baptista2021angular, leite2021observation,baptista2021searches,collaboration2021measurement,baptista2021measurement,aaij2021precision,aaij2022identification,bediaga2020measurement, aaij2021constraints,   onderwater2020study, aaij2022arxiv,     aaij2019arxiv,aaij2021evidence,aaij2020aps,baptista2021observation,baptista2021search,aaij2020isospin,  aaij2019precision,aaij2019search, lhcb2108evidence,aaij2022study,         aaij2022first,aaij2022j,  aaij2022observation, aaij2022tests}.


{\contribution} Автор принимал активное участие в разработке модели поддержки прнятия решений, активно разрабатывал и анализировал архитектуры программного обеспечения для автоматизированных систем управления. 

\ifnumequal{\value{bibliosel}}{0}
{%%% Встроенная реализация с загрузкой файла через движок bibtex8. (При желании, внутри можно использовать обычные ссылки, наподобие `\cite{vakbib1,vakbib2}`).
    {\publications} Основные результаты по теме диссертации изложены
    в 2 печатных изданиях,
    2 из которых изданы в журналах, рекомендованных ВАК,
    1 в тезисах докладов.
}%
{%%% Реализация пакетом biblatex через движок biber
    \begin{refsection}[bl-author, bl-registered]
        % Это refsection=1.
        % Процитированные здесь работы:
        %  * подсчитываются, для автоматического составления фразы "Основные результаты ..."
        %  * попадают в авторскую библиографию, при usefootcite==0 и стиле `\insertbiblioauthor` или `\insertbiblioauthorgrouped`
        %  * нумеруются там в зависимости от порядка команд `\printbibliography` в этом разделе.
        %  * при использовании `\insertbiblioauthorgrouped`, порядок команд `\printbibliography` в нём должен быть тем же (см. biblio/biblatex.tex)
        %
        % Невидимый библиографический список для подсчёта количества публикаций:
        \printbibliography[heading=nobibheading, section=1, env=countauthorvak,          keyword=biblioauthorvak]%
        \printbibliography[heading=nobibheading, section=1, env=countauthorwos,          keyword=biblioauthorwos]%
        \printbibliography[heading=nobibheading, section=1, env=countauthorscopus,       keyword=biblioauthorscopus]%
        \printbibliography[heading=nobibheading, section=1, env=countauthorconf,         keyword=biblioauthorconf]%
        \printbibliography[heading=nobibheading, section=1, env=countauthorother,        keyword=biblioauthorother]%
        \printbibliography[heading=nobibheading, section=1, env=countregistered,         keyword=biblioregistered]%
        \printbibliography[heading=nobibheading, section=1, env=countauthorpatent,       keyword=biblioauthorpatent]%
        \printbibliography[heading=nobibheading, section=1, env=countauthorprogram,      keyword=biblioauthorprogram]%
        \printbibliography[heading=nobibheading, section=1, env=countauthor,             keyword=biblioauthor]%
        \printbibliography[heading=nobibheading, section=1, env=countauthorvakscopuswos, filter=vakscopuswos]%
        \printbibliography[heading=nobibheading, section=1, env=countauthorscopuswos,    filter=scopuswos]%
        %
        \nocite{*}%
        %
        {\publications} Основные результаты по теме диссертации изложены в~\cite{aaij2022searches, baptista2021angular, leite2021observation,baptista2021searches,collaboration2021measurement,baptista2021measurement,aaij2021precision,aaij2022identification,bediaga2020measurement, aaij2021constraints,   onderwater2020study, aaij2022arxiv,     aaij2019arxiv,aaij2021evidence,aaij2020aps,baptista2021observation,baptista2021search,aaij2020isospin,  aaij2019precision,aaij2019search, lhcb2108evidence,aaij2022study,         aaij2022first,aaij2022j,  aaij2022observation, aaij2022tests}~печатных изданиях,
        \arabic{citeauthorvak} 2 из которых изданы в журналах, рекомендованных ВАК\sloppy%
        \ifnum \value{citeauthorscopuswos}>0%
            , \arabic{citeauthorscopuswos} ~\cite{aaij2022searches, baptista2021angular, leite2021observation,baptista2021searches,collaboration2021measurement,baptista2021measurement,aaij2021precision,aaij2022identification,bediaga2020measurement, aaij2021constraints,   onderwater2020study, aaij2022arxiv,     aaij2019arxiv,aaij2021evidence,aaij2020aps,baptista2021observation,baptista2021search,aaij2020isospin,  aaij2019precision,aaij2019search, lhcb2108evidence,aaij2022study,         aaij2022first,aaij2022j,  aaij2022observation, aaij2022tests} в~периодических научных журналах, индексируемых Web of~Science и Scopus\sloppy%
        \fi%
        \ifnum \value{citeauthorconf}>0%
            , \arabic{citeauthorconf} "--- в~тезисах докладов.
        \else%
            .
        \fi%
        \ifnum \value{citeregistered}=1%
            \ifnum \value{citeauthorpatent}=1%
                Зарегистрирован \arabic{citeauthorpatent} патент.
            \fi%
            \ifnum \value{citeauthorprogram}=1%
                Зарегистрирована \arabic{citeauthorprogram} программа для ЭВМ.
            \fi%
        \fi%
        \ifnum \value{citeregistered}>1%
            Зарегистрированы\ %
            \ifnum \value{citeauthorpatent}>0%
            \formbytotal{citeauthorpatent}{патент}{}{а}{}\sloppy%
            \ifnum \value{citeauthorprogram}=0 . \else \ и~\fi%
            \fi%
            \ifnum \value{citeauthorprogram}>0%
            \formbytotal{citeauthorprogram}{программ}{а}{ы}{} для ЭВМ.
            \fi%
        \fi%
        % К публикациям, в которых излагаются основные научные результаты диссертации на соискание учёной
        % степени, в рецензируемых изданиях приравниваются патенты на изобретения, патенты (свидетельства) на
        % полезную модель, патенты на промышленный образец, патенты на селекционные достижения, свидетельства
        % на программу для электронных вычислительных машин, базу данных, топологию интегральных микросхем,
        % зарегистрированные в установленном порядке.(в ред. Постановления Правительства РФ от 21.04.2016 N 335)
    \end{refsection}%
    \begin{refsection}[bl-author, bl-registered]
        % Это refsection=2.
        % Процитированные здесь работы:
        %  * попадают в авторскую библиографию, при usefootcite==0 и стиле `\insertbiblioauthorimportant`.
        %  * ни на что не влияют в противном случае
        \nocite{vakbib2}%vak
        \nocite{patbib1}%patent
        \nocite{progbib1}%program
        \nocite{bib1}%other
        \nocite{confbib1}%conf
    \end{refsection}%
        %
        % Всё, что вне этих двух refsection, это refsection=0,
        %  * для диссертации - это нормальные ссылки, попадающие в обычную библиографию
        %  * для автореферата:
        %     * при usefootcite==0, ссылка корректно сработает только для источника из `external.bib`. Для своих работ --- напечатает "[0]" (и даже Warning не вылезет).
        %     * при usefootcite==1, ссылка сработает нормально. В авторской библиографии будут только процитированные в refsection=0 работы.
}
