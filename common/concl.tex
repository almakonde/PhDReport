%% Согласно ГОСТ Р 7.0.11-2011:
%% 5.3.3 В заключении диссертации излагают итоги выполненного исследования, рекомендации, перспективы дальнейшей разработки темы.
%% 9.2.3 В заключении автореферата диссертации излагают итоги данного исследования, рекомендации и перспективы дальнейшей разработки темы.

Итоги исследования:
\begin{enumerate}
  \item На основе анализа работы нейронных сетей, экспертных систем разработана модель поддержки принятия решений о проектировании архитектуры различных типов автоиматизированных систем управления,
  \item Численные исследования показали, что модель устойчива при малых отклонениях в возмущениях, т.е. недообучение и переобучение просиходят только в условиях потери устойчивости работы нейронной сети и нечеткого логического вывода в условиях сильных отклонений при обучении,
  \item Математическое моделирование показало, что модель выявляет эвристические соответствия между лингвистическими термами на основе аппроксимации и функционального отоборажения принадлежности из семантического пространства в численное за счет поправки ошибки и вектора градиента, выступающего как адаптивная система поправки ошибки,
  \item Для выполнения поставленных задач был создан алгоритм системы поддержки принятия решений, проанализирован и структурирован набор пользовательских запросов для формализации типа информационной системы, выполнена работа по проектировани. и реализации системы поддержки принятия решений.
\end{enumerate}

Рекомендации:
\begin{enumerate}
  \item в ходе исследования не получилось спроектировать систему организации шаблонов проектирования кода программного обеспечения, т.к. существует бесчисленный набор комбинаторных комбинаций по организации сущностей в логике программного обеспечения, 
  \item логика организации внутренней иерархии сущностей крайне зависит от конкретной доменной области технологического внедрения программного обеспечения, что делает невозможным предположить некоторую типовую организацию сущностей,
  \item логика иерархии таблиц в базах данных такэе подвержена многочисленным вероятным возможным вариантам связывания отношениями между собой и аналогичтным образом определяется доменной структурой технологического внедрения программного обеспечния, что делает невозможным предугадать рациональную декомпозицию, без знаний целей проектирования системы.
\end{enumerate}

Таким образом, работу по проектированию модели поддержки принятия решений была направлена на решение проблем с проектированием архитектуры программного обеспечения, т.к. в этой области есть сформированные типовые наборы архитектур внутренней и внещней организации для типовых классов автоматизированных систем управления.

Перспективы такого направления включают увеличение базы знаний, обучение за счет пользователей системы, размещение модели во всеобзем доступе, сбор обратной информации, улучшение функционала отображения входных переменных в выходные.